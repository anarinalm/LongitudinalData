% Options for packages loaded elsewhere
\PassOptionsToPackage{unicode}{hyperref}
\PassOptionsToPackage{hyphens}{url}
\PassOptionsToPackage{dvipsnames,svgnames,x11names}{xcolor}
%
\documentclass[
  letterpaper,
  DIV=11,
  numbers=noendperiod]{scrreprt}

\usepackage{amsmath,amssymb}
\usepackage{iftex}
\ifPDFTeX
  \usepackage[T1]{fontenc}
  \usepackage[utf8]{inputenc}
  \usepackage{textcomp} % provide euro and other symbols
\else % if luatex or xetex
  \usepackage{unicode-math}
  \defaultfontfeatures{Scale=MatchLowercase}
  \defaultfontfeatures[\rmfamily]{Ligatures=TeX,Scale=1}
\fi
\usepackage{lmodern}
\ifPDFTeX\else  
    % xetex/luatex font selection
\fi
% Use upquote if available, for straight quotes in verbatim environments
\IfFileExists{upquote.sty}{\usepackage{upquote}}{}
\IfFileExists{microtype.sty}{% use microtype if available
  \usepackage[]{microtype}
  \UseMicrotypeSet[protrusion]{basicmath} % disable protrusion for tt fonts
}{}
\makeatletter
\@ifundefined{KOMAClassName}{% if non-KOMA class
  \IfFileExists{parskip.sty}{%
    \usepackage{parskip}
  }{% else
    \setlength{\parindent}{0pt}
    \setlength{\parskip}{6pt plus 2pt minus 1pt}}
}{% if KOMA class
  \KOMAoptions{parskip=half}}
\makeatother
\usepackage{xcolor}
\setlength{\emergencystretch}{3em} % prevent overfull lines
\setcounter{secnumdepth}{-\maxdimen} % remove section numbering
% Make \paragraph and \subparagraph free-standing
\ifx\paragraph\undefined\else
  \let\oldparagraph\paragraph
  \renewcommand{\paragraph}[1]{\oldparagraph{#1}\mbox{}}
\fi
\ifx\subparagraph\undefined\else
  \let\oldsubparagraph\subparagraph
  \renewcommand{\subparagraph}[1]{\oldsubparagraph{#1}\mbox{}}
\fi

\usepackage{color}
\usepackage{fancyvrb}
\newcommand{\VerbBar}{|}
\newcommand{\VERB}{\Verb[commandchars=\\\{\}]}
\DefineVerbatimEnvironment{Highlighting}{Verbatim}{commandchars=\\\{\}}
% Add ',fontsize=\small' for more characters per line
\usepackage{framed}
\definecolor{shadecolor}{RGB}{241,243,245}
\newenvironment{Shaded}{\begin{snugshade}}{\end{snugshade}}
\newcommand{\AlertTok}[1]{\textcolor[rgb]{0.68,0.00,0.00}{#1}}
\newcommand{\AnnotationTok}[1]{\textcolor[rgb]{0.37,0.37,0.37}{#1}}
\newcommand{\AttributeTok}[1]{\textcolor[rgb]{0.40,0.45,0.13}{#1}}
\newcommand{\BaseNTok}[1]{\textcolor[rgb]{0.68,0.00,0.00}{#1}}
\newcommand{\BuiltInTok}[1]{\textcolor[rgb]{0.00,0.23,0.31}{#1}}
\newcommand{\CharTok}[1]{\textcolor[rgb]{0.13,0.47,0.30}{#1}}
\newcommand{\CommentTok}[1]{\textcolor[rgb]{0.37,0.37,0.37}{#1}}
\newcommand{\CommentVarTok}[1]{\textcolor[rgb]{0.37,0.37,0.37}{\textit{#1}}}
\newcommand{\ConstantTok}[1]{\textcolor[rgb]{0.56,0.35,0.01}{#1}}
\newcommand{\ControlFlowTok}[1]{\textcolor[rgb]{0.00,0.23,0.31}{#1}}
\newcommand{\DataTypeTok}[1]{\textcolor[rgb]{0.68,0.00,0.00}{#1}}
\newcommand{\DecValTok}[1]{\textcolor[rgb]{0.68,0.00,0.00}{#1}}
\newcommand{\DocumentationTok}[1]{\textcolor[rgb]{0.37,0.37,0.37}{\textit{#1}}}
\newcommand{\ErrorTok}[1]{\textcolor[rgb]{0.68,0.00,0.00}{#1}}
\newcommand{\ExtensionTok}[1]{\textcolor[rgb]{0.00,0.23,0.31}{#1}}
\newcommand{\FloatTok}[1]{\textcolor[rgb]{0.68,0.00,0.00}{#1}}
\newcommand{\FunctionTok}[1]{\textcolor[rgb]{0.28,0.35,0.67}{#1}}
\newcommand{\ImportTok}[1]{\textcolor[rgb]{0.00,0.46,0.62}{#1}}
\newcommand{\InformationTok}[1]{\textcolor[rgb]{0.37,0.37,0.37}{#1}}
\newcommand{\KeywordTok}[1]{\textcolor[rgb]{0.00,0.23,0.31}{#1}}
\newcommand{\NormalTok}[1]{\textcolor[rgb]{0.00,0.23,0.31}{#1}}
\newcommand{\OperatorTok}[1]{\textcolor[rgb]{0.37,0.37,0.37}{#1}}
\newcommand{\OtherTok}[1]{\textcolor[rgb]{0.00,0.23,0.31}{#1}}
\newcommand{\PreprocessorTok}[1]{\textcolor[rgb]{0.68,0.00,0.00}{#1}}
\newcommand{\RegionMarkerTok}[1]{\textcolor[rgb]{0.00,0.23,0.31}{#1}}
\newcommand{\SpecialCharTok}[1]{\textcolor[rgb]{0.37,0.37,0.37}{#1}}
\newcommand{\SpecialStringTok}[1]{\textcolor[rgb]{0.13,0.47,0.30}{#1}}
\newcommand{\StringTok}[1]{\textcolor[rgb]{0.13,0.47,0.30}{#1}}
\newcommand{\VariableTok}[1]{\textcolor[rgb]{0.07,0.07,0.07}{#1}}
\newcommand{\VerbatimStringTok}[1]{\textcolor[rgb]{0.13,0.47,0.30}{#1}}
\newcommand{\WarningTok}[1]{\textcolor[rgb]{0.37,0.37,0.37}{\textit{#1}}}

\providecommand{\tightlist}{%
  \setlength{\itemsep}{0pt}\setlength{\parskip}{0pt}}\usepackage{longtable,booktabs,array}
\usepackage{calc} % for calculating minipage widths
% Correct order of tables after \paragraph or \subparagraph
\usepackage{etoolbox}
\makeatletter
\patchcmd\longtable{\par}{\if@noskipsec\mbox{}\fi\par}{}{}
\makeatother
% Allow footnotes in longtable head/foot
\IfFileExists{footnotehyper.sty}{\usepackage{footnotehyper}}{\usepackage{footnote}}
\makesavenoteenv{longtable}
\usepackage{graphicx}
\makeatletter
\def\maxwidth{\ifdim\Gin@nat@width>\linewidth\linewidth\else\Gin@nat@width\fi}
\def\maxheight{\ifdim\Gin@nat@height>\textheight\textheight\else\Gin@nat@height\fi}
\makeatother
% Scale images if necessary, so that they will not overflow the page
% margins by default, and it is still possible to overwrite the defaults
% using explicit options in \includegraphics[width, height, ...]{}
\setkeys{Gin}{width=\maxwidth,height=\maxheight,keepaspectratio}
% Set default figure placement to htbp
\makeatletter
\def\fps@figure{htbp}
\makeatother

\usepackage{booktabs}
\usepackage{longtable}
\usepackage{array}
\usepackage{multirow}
\usepackage{wrapfig}
\usepackage{float}
\usepackage{colortbl}
\usepackage{pdflscape}
\usepackage{tabu}
\usepackage{threeparttable}
\usepackage{threeparttablex}
\usepackage[normalem]{ulem}
\usepackage{makecell}
\usepackage{xcolor}
\usepackage{makeidx}
\makeindex
\KOMAoption{captions}{tableheading}
\makeatletter
\makeatother
\makeatletter
\makeatother
\makeatletter
\@ifpackageloaded{caption}{}{\usepackage{caption}}
\AtBeginDocument{%
\ifdefined\contentsname
  \renewcommand*\contentsname{Table of contents}
\else
  \newcommand\contentsname{Table of contents}
\fi
\ifdefined\listfigurename
  \renewcommand*\listfigurename{List of Figures}
\else
  \newcommand\listfigurename{List of Figures}
\fi
\ifdefined\listtablename
  \renewcommand*\listtablename{List of Tables}
\else
  \newcommand\listtablename{List of Tables}
\fi
\ifdefined\figurename
  \renewcommand*\figurename{Figure}
\else
  \newcommand\figurename{Figure}
\fi
\ifdefined\tablename
  \renewcommand*\tablename{Table}
\else
  \newcommand\tablename{Table}
\fi
}
\@ifpackageloaded{float}{}{\usepackage{float}}
\floatstyle{ruled}
\@ifundefined{c@chapter}{\newfloat{codelisting}{h}{lop}}{\newfloat{codelisting}{h}{lop}[chapter]}
\floatname{codelisting}{Listing}
\newcommand*\listoflistings{\listof{codelisting}{List of Listings}}
\makeatother
\makeatletter
\@ifpackageloaded{caption}{}{\usepackage{caption}}
\@ifpackageloaded{subcaption}{}{\usepackage{subcaption}}
\makeatother
\makeatletter
\@ifpackageloaded{tcolorbox}{}{\usepackage[skins,breakable]{tcolorbox}}
\makeatother
\makeatletter
\@ifundefined{shadecolor}{\definecolor{shadecolor}{rgb}{.97, .97, .97}}
\makeatother
\makeatletter
\makeatother
\makeatletter
\makeatother
\ifLuaTeX
  \usepackage{selnolig}  % disable illegal ligatures
\fi
\IfFileExists{bookmark.sty}{\usepackage{bookmark}}{\usepackage{hyperref}}
\IfFileExists{xurl.sty}{\usepackage{xurl}}{} % add URL line breaks if available
\urlstyle{same} % disable monospaced font for URLs
\hypersetup{
  colorlinks=true,
  linkcolor={blue},
  filecolor={Maroon},
  citecolor={Blue},
  urlcolor={Blue},
  pdfcreator={LaTeX via pandoc}}

\author{}
\date{}

\begin{document}
\ifdefined\Shaded\renewenvironment{Shaded}{\begin{tcolorbox}[boxrule=0pt, enhanced, interior hidden, sharp corners, breakable, borderline west={3pt}{0pt}{shadecolor}, frame hidden]}{\end{tcolorbox}}\fi

\hypertarget{sec-longi-intro}{%
\chapter*{Longitudinal Data Analysis}\label{sec-longi-intro}}
\addcontentsline{toc}{chapter}{Longitudinal Data Analysis}

\hypertarget{data-structure}{%
\section{Data Structure}\label{data-structure}}

\begin{Shaded}
\begin{Highlighting}[]
\FunctionTok{suppressPackageStartupMessages}\NormalTok{(}\FunctionTok{library}\NormalTok{(tidyr))}
\FunctionTok{suppressPackageStartupMessages}\NormalTok{(}\FunctionTok{library}\NormalTok{(data.table))}
\FunctionTok{suppressPackageStartupMessages}\NormalTok{(}\FunctionTok{library}\NormalTok{(ggplot2))}
\FunctionTok{suppressPackageStartupMessages}\NormalTok{(}\FunctionTok{library}\NormalTok{(haven))}
\FunctionTok{suppressPackageStartupMessages}\NormalTok{(}\FunctionTok{library}\NormalTok{(dplyr))}
\FunctionTok{suppressPackageStartupMessages}\NormalTok{(}\FunctionTok{library}\NormalTok{(GGally))}
\FunctionTok{suppressPackageStartupMessages}\NormalTok{(}\FunctionTok{library}\NormalTok{(kableExtra))}
\end{Highlighting}
\end{Shaded}

The most important part of any statistical analysis begins with loading
the data into Rstudio. Data can come in many forms with two popular ones
being csv (comma separated values) and dta. Below we show different
methods for how to load the data into RStudio.

\hypertarget{loading-csv-files}{%
\subsection{Loading CSV files}\label{loading-csv-files}}

\hypertarget{using-base-r}{%
\subsubsection{Using base R}\label{using-base-r}}

The following method is a pretty standard way of loading csv files into
R. It requires no external packages (this is already a base R function)
and works as follows. First, specify the location of your data, and put
it into function as an input.

\begin{Shaded}
\begin{Highlighting}[]
\NormalTok{TLC }\OtherTok{\textless{}{-}} \FunctionTok{read.csv}\NormalTok{(}\StringTok{"Data/TLC.csv"}\NormalTok{)}
\end{Highlighting}
\end{Shaded}

We can then get a look at the data by using the head function which
provides us with a sneak peek of the first n rows.

\begin{Shaded}
\begin{Highlighting}[]
\FunctionTok{head}\NormalTok{(TLC, }\AttributeTok{n =} \DecValTok{10}\NormalTok{)}
\end{Highlighting}
\end{Shaded}

\begin{verbatim}
   id lead0 lead1 lead4 lead6     group
1   1  30.8  26.9  25.8  23.8   Placebo
2   2  26.5  14.8  19.5  21.0 Treatment
3   3  25.8  23.0  19.1  23.2 Treatment
4   4  24.7  24.5  22.0  22.5   Placebo
5   5  20.4   2.8   3.2   9.4 Treatment
6   6  20.4   5.4   4.5  11.9 Treatment
7   7  28.6  20.8  19.2  18.4   Placebo
8   8  33.7  31.6  28.5  25.1   Placebo
9   9  19.7  14.9  15.3  14.7   Placebo
10 10  31.1  31.2  29.2  30.1   Placebo
\end{verbatim}

\hypertarget{using-the-readr-package}{%
\subsubsection{Using the readr package}\label{using-the-readr-package}}

The next method requires the use of the readr package. It works exactly
the same as read.csv, save for the fact that it is faster than read.csv.

\begin{Shaded}
\begin{Highlighting}[]
\FunctionTok{library}\NormalTok{(readr)}
\NormalTok{TLC }\OtherTok{\textless{}{-}} \FunctionTok{read\_csv}\NormalTok{(}\StringTok{"Data/TLC.csv"}\NormalTok{)}
\end{Highlighting}
\end{Shaded}

We can also print the first few rows to take a look of our data using
function \texttt{head}, here we print the first 10 rows of the data.

\begin{Shaded}
\begin{Highlighting}[]
\FunctionTok{head}\NormalTok{(TLC, }\AttributeTok{n =} \DecValTok{10}\NormalTok{)}
\end{Highlighting}
\end{Shaded}

\begin{verbatim}
# A tibble: 10 x 6
      id lead0 lead1 lead4 lead6 group    
   <dbl> <dbl> <dbl> <dbl> <dbl> <chr>    
 1     1  30.8 26.9   25.8 23.8  Placebo  
 2     2  26.5 14.8   19.5 21    Treatment
 3     3  25.8 23     19.1 23.2  Treatment
 4     4  24.7 24.5   22   22.5  Placebo  
 5     5  20.4  2.8    3.2  9.40 Treatment
 6     6  20.4  5.40   4.5 11.9  Treatment
 7     7  28.6 20.8   19.2 18.4  Placebo  
 8     8  33.7 31.6   28.5 25.1  Placebo  
 9     9  19.7 14.9   15.3 14.7  Placebo  
10    10  31.1 31.2   29.2 30.1  Placebo  
\end{verbatim}

\hypertarget{using-the-data.table-package}{%
\subsubsection{Using the data.table
package}\label{using-the-data.table-package}}

If we have large datasets, we can use the fread function in the
data.table package to read the data faster compared to the other methods
above, and we print the first 5 rows of the data.

\begin{Shaded}
\begin{Highlighting}[]
\FunctionTok{library}\NormalTok{(data.table)}
\NormalTok{TLC }\OtherTok{\textless{}{-}} \FunctionTok{fread}\NormalTok{(}\StringTok{"Data/TLC.csv"}\NormalTok{)}
\FunctionTok{head}\NormalTok{(TLC, }\AttributeTok{n =} \DecValTok{5}\NormalTok{)}
\end{Highlighting}
\end{Shaded}

\begin{verbatim}
   id lead0 lead1 lead4 lead6     group
1:  1  30.8  26.9  25.8  23.8   Placebo
2:  2  26.5  14.8  19.5  21.0 Treatment
3:  3  25.8  23.0  19.1  23.2 Treatment
4:  4  24.7  24.5  22.0  22.5   Placebo
5:  5  20.4   2.8   3.2   9.4 Treatment
\end{verbatim}

\hypertarget{loading-dta-files}{%
\subsection{Loading dta files}\label{loading-dta-files}}

We can also read files in other formats from other software (STATA,
SPSS, SAS, etc). Here we will explore reading dta files which is used in
STATA software. In order to load these into Rstudio we need to use a
package known as haven. The haven package has a function known as
\texttt{read\_dta()} which serves a similar purpose as
\texttt{read.csv()}, \texttt{read\_csv()} and \texttt{fread()}.

\begin{Shaded}
\begin{Highlighting}[]
\NormalTok{TLCdta }\OtherTok{\textless{}{-}} \FunctionTok{read\_dta}\NormalTok{(}\StringTok{"Data/TLC.dta"}\NormalTok{)}
\FunctionTok{head}\NormalTok{(TLCdta, }\AttributeTok{n =} \DecValTok{15}\NormalTok{)}
\end{Highlighting}
\end{Shaded}

\begin{verbatim}
# A tibble: 15 x 6
      id lead0 lead1 lead4 lead6 group    
   <dbl> <dbl> <dbl> <dbl> <dbl> <chr>    
 1     1  30.8 26.9  25.8  23.8  Placebo  
 2     2  26.5 14.8  19.5  21    Treatment
 3     3  25.8 23    19.1  23.2  Treatment
 4     4  24.7 24.5  22    22.5  Placebo  
 5     5  20.4  2.80  3.20  9.40 Treatment
 6     6  20.4  5.40  4.5  11.9  Treatment
 7     7  28.6 20.8  19.2  18.4  Placebo  
 8     8  33.7 31.6  28.5  25.1  Placebo  
 9     9  19.7 14.9  15.3  14.7  Placebo  
10    10  31.1 31.2  29.2  30.1  Placebo  
11    11  19.8 17.5  20.5  27.5  Placebo  
12    12  24.8 23.1  24.6  30.9  Treatment
13    13  21.4 26.3  19.5  19    Placebo  
14    14  27.9  6.30 18.5  16.3  Treatment
15    15  21.1 20.3  18.4  20.8  Placebo  
\end{verbatim}



\printindex

\end{document}
